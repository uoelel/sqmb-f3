% Options for packages loaded elsewhere
\PassOptionsToPackage{unicode}{hyperref}
\PassOptionsToPackage{hyphens}{url}
%
\documentclass[
]{article}
\usepackage{amsmath,amssymb}
\usepackage{lmodern}
\usepackage{iftex}
\ifPDFTeX
  \usepackage[T1]{fontenc}
  \usepackage[utf8]{inputenc}
  \usepackage{textcomp} % provide euro and other symbols
\else % if luatex or xetex
  \usepackage{unicode-math}
  \defaultfontfeatures{Scale=MatchLowercase}
  \defaultfontfeatures[\rmfamily]{Ligatures=TeX,Scale=1}
  \setmainfont[]{Arial}
\fi
% Use upquote if available, for straight quotes in verbatim environments
\IfFileExists{upquote.sty}{\usepackage{upquote}}{}
\IfFileExists{microtype.sty}{% use microtype if available
  \usepackage[]{microtype}
  \UseMicrotypeSet[protrusion]{basicmath} % disable protrusion for tt fonts
}{}
\makeatletter
\@ifundefined{KOMAClassName}{% if non-KOMA class
  \IfFileExists{parskip.sty}{%
    \usepackage{parskip}
  }{% else
    \setlength{\parindent}{0pt}
    \setlength{\parskip}{6pt plus 2pt minus 1pt}}
}{% if KOMA class
  \KOMAoptions{parskip=half}}
\makeatother
\usepackage{xcolor}
\usepackage[margin=1in]{geometry}
\usepackage{graphicx}
\makeatletter
\def\maxwidth{\ifdim\Gin@nat@width>\linewidth\linewidth\else\Gin@nat@width\fi}
\def\maxheight{\ifdim\Gin@nat@height>\textheight\textheight\else\Gin@nat@height\fi}
\makeatother
% Scale images if necessary, so that they will not overflow the page
% margins by default, and it is still possible to overwrite the defaults
% using explicit options in \includegraphics[width, height, ...]{}
\setkeys{Gin}{width=\maxwidth,height=\maxheight,keepaspectratio}
% Set default figure placement to htbp
\makeatletter
\def\fps@figure{htbp}
\makeatother
\setlength{\emergencystretch}{3em} % prevent overfull lines
\providecommand{\tightlist}{%
  \setlength{\itemsep}{0pt}\setlength{\parskip}{0pt}}
\setcounter{secnumdepth}{5}
\ifLuaTeX
  \usepackage{selnolig}  % disable illegal ligatures
\fi
\IfFileExists{bookmark.sty}{\usepackage{bookmark}}{\usepackage{hyperref}}
\IfFileExists{xurl.sty}{\usepackage{xurl}}{} % add URL line breaks if available
\urlstyle{same} % disable monospaced font for URLs
\hypersetup{
  pdftitle={SQMB 2022/23 Formative Assessment F3},
  pdfauthor={EXAM NUMBER},
  hidelinks,
  pdfcreator={LaTeX via pandoc}}

\title{SQMB 2022/23 Formative Assessment F3}
\author{EXAM NUMBER}
\date{2023-03-13}

\begin{document}
\maketitle

\textbf{DEADLINE: THURSDAY 6 APRIL AT NOON}

\hypertarget{overview}{%
\section{Overview}\label{overview}}

In this assessment, you'll be working with corpus data on the English
dative alternation. In English, a number of ditransitive verbs (e.g.,
\emph{give}, \emph{feed}, \emph{teach}) have two different ways to
express the recipient of the action:

\begin{itemize}
\tightlist
\item
  as a \textbf{prepositional phrase (PP)}: The child gives a treat
  \emph{to the dog}.
\item
  as a \textbf{noun phrase (NP)}: The child gives \emph{the dog} a
  treat.
\end{itemize}

In these examples, \emph{the child} is the giver, \emph{the dog} is the
recipient, and \emph{a treat} is the theme.

The data you'll be analysing can be found in \texttt{data/dative.csv}.
It comes from \href{https://web.stanford.edu/~bresnan/qs-submit.pdf}{a
2005 study by Joan Bresnan and colleagues}, and it was gathered from the
Switchboard corpus and the Treebank Wall Street Journal collection.

The data frame contains the following columns:

\begin{itemize}
\tightlist
\item
  \texttt{RealizationRec}: Whether the sentence's recipient is realised
  as a noun phrase (NP) or a prepositional phrase (PP).
\item
  \texttt{Verb}: The verb in the sentence.
\item
  \texttt{AnimacyOfRec}: Animacy of the recipient (animate
  vs.~inanimate).
\item
  \texttt{AnimacyOfTheme}: Animacy of the theme (animate vs.~inanimate).
\item
  \texttt{LengthOfTheme}: Number of words that make up the theme.
\end{itemize}

Your task is to answer the following research question: \textbf{What
roles do the animacy of the theme and the length of the theme play for
the realisation of the recipient?}

Some pointers:

\begin{itemize}
\item
  \textbf{Think about the variables that'll go into the analysis:} Which
  is the outcome variable? What kind of variable is it? And which are
  the predictors? What kinds of variable are they?
\item
  \textbf{Summarise the relevant data:} Pick appropriate summary
  measures depending on the nature of the variable (mean, median, mode;
  standard deviation, range).
\item
  \textbf{Plot the data:} Use appropriate plots to visualise patterns in
  the data that are relevant to the research question above. Concisely
  describe the patterns you see.
\item
  \textbf{Prepare the data for analysis:} How will you set up your
  outcome variable? How will categorical predictors be coded/ordered?
  What will this mean for their interpretation? Will you transform
  numeric predictors, and if yes, how?
\item
  \textbf{Fit a model to the data:} What distribution family should you
  use? What model formula will you use? (Optional: What does the
  mathematical model specification look like?)
\item
  \textbf{Report the model and its estimates:} Describe the model that
  you fit. Summarise its estimates/posteriors in writing and using
  plots.
\item
  \textbf{Address the research question:} Interpret the model's
  estimates. What do they mean for the research question?
\end{itemize}

When you are satisfied, you can \textbf{submit your project} by:

\begin{enumerate}
\def\labelenumi{\arabic{enumi}.}
\tightlist
\item
  rendering your Rmd file to PDF, and
\item
  uploading it to Turnitin via Learn.
\end{enumerate}

\end{document}
